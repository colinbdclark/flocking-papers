% -----------------------------------------------
% Template for ICMC SMC 2014
% adapted and corrected from the template for SMC 2013,  which was adapted from that of  SMC 2012, which was adapted from that of SMC 2011
% -----------------------------------------------

\documentclass{article}
\usepackage{icmcsmc2014}
\usepackage{times}
\usepackage{ifpdf}
\usepackage[english]{babel}
%\usepackage{cite}

%%%%%%%%%%%%%%%%%%%%%%%% Some useful packages %%%%%%%%%%%%%%%%%%%%%%%%%%%%%%%
%%%%%%%%%%%%%%%%%%%%%%%% See related documentation %%%%%%%%%%%%%%%%%%%%%%%%%%
%\usepackage{amsmath} % popular packages from Am. Math. Soc. Please use the
%\usepackage{amssymb} % related math environments (split, subequation, cases,
%\usepackage{amsfonts}% multline, etc.)
%\usepackage{bm}      % Bold Math package, defines the command \bf{}
%\usepackage{paralist}% extended list environments
%%subfig.sty is the modern replacement for subfigure.sty. However, subfig.sty
%%requires and automatically loads caption.sty which overrides class handling
%%of captions. To prevent this problem, preload caption.sty with caption=false
%\usepackage[caption=false]{caption}
%\usepackage[font=footnotesize]{subfig}


%user defined variables
\def\papertitle{Flocking.js: An efficient declarative web based programming language}
\def\firstauthor{Colin Clark}
\def\secondauthor{Adam Tindale}
\def\thirdauthor{nobody}
% adds the automatic
% Saves a lot of ouptut space in PDF... after conversion with the distiller
% Delete if you cannot get PS fonts working on your system.

% pdf-tex settings: detect automatically if run by latex or pdflatex
\newif\ifpdf
\ifx\pdfoutput\relax
\else
   \ifcase\pdfoutput
      \pdffalse
   \else
      \pdftrue
\fi

\ifpdf % compiling with pdflatex
  \usepackage[pdftex,
    pdftitle={\papertitle},
    pdfauthor={\firstauthor, \secondauthor},
    bookmarksnumbered, % use section numbers with bookmarks
    pdfstartview=XYZ % start with zoom=100% instead of full screen;
                     % especially useful if working with a big screen :-)
   ]{hyperref}
  %\pdfcompresslevel=9

  \usepackage[pdftex]{graphicx}
  % declare the path(s) where your graphic files are and their extensions so
  %you won't have to specify these with every instance of \includegraphics
  \graphicspath{{./figures/}}
  \DeclareGraphicsExtensions{.pdf,.jpeg,.png}

  \usepackage[figure,table]{hypcap}

\else % compiling with latex
  \usepackage[dvips,
    bookmarksnumbered, % use section numbers with bookmarks
    pdfstartview=XYZ % start with zoom=100% instead of full screen
  ]{hyperref}  % hyperrefs are active in the pdf file after conversion

  \usepackage[dvips]{epsfig,graphicx}
  % declare the path(s) where your graphic files are and their extensions so
  %you won't have to specify these with every instance of \includegraphics
  \graphicspath{{./figures/}}
  \DeclareGraphicsExtensions{.eps}

  \usepackage[figure,table]{hypcap}
\fi

%setup the hyperref package - make the links black without a surrounding frame
\hypersetup{
    colorlinks,%
    citecolor=black,%
    filecolor=black,%
    linkcolor=black,%
    urlcolor=black
}


% Title.
% ------
\title{\papertitle}

% Authors
% Please note that submissions are NOT anonymous, therefore
% authors' names have to be VISIBLE in your manuscript.
%
% Single address
% To use with only one author or several with the same address
% ---------------
%\oneauthor
%   {\firstauthor} {Affiliation1 \\ %
%     {\tt \href{mailto:author1@smcnetwork.org}{author1@smcnetwork.org}}}

%Two addresses
%--------------
 \twoauthors
   {\firstauthor} {OCAD University \\ %
     {\tt \href{mailto:cclark@ocadu.ca}{cclark@ocadu.ca}}}
   {\secondauthor} {OCAD University \\ %
     {\tt \href{mailto:atindale@faculty.ocadu.ca}{atindale@faculty.ocadu.ca}}}

% Three addresses
% --------------
% \threeauthors
%   {\firstauthor} {Affiliation1 \\ %
%     {\tt \href{mailto:author1@smcnetwork.org}{author1@smcnetwork.org}}}
%   {\secondauthor} {Affiliation2 \\ %
%     {\tt \href{mailto:author2@smcnetwork.org}{author2@smcnetwork.org}}}
%   {\thirdauthor} { Affiliation3 \\ %
%     {\tt \href{mailto:author3@smcnetwork.org}{author3@smcnetwork.org}}}


% ***************************************** the document starts here ***************
\begin{document}
%
\capstartfalse
\maketitle
\capstarttrue
%
\begin{abstract}
Flocking\footnote{http://flockingjs.org/} is a framework for audio synthesis and music composition written in JavaScript. It takes a unique approach to solving several of the typical architectural problems faced by computer music environments, emphasizing a declarative style that is closely aligned with the principles of the web.

In Flocking, instruments and scores are defined as JavaScript Object Notation (JSON) objects. JSON is a subset of the JavaScript language that is used widely across the web for exchanging data. By representing the basic building blocks of synthesis declaratively, Flocking’s goal is to enable the growth of an ecosystem of tools that can easily parse and understand the logic and semantics of a digital instrument. This is particularly useful for supporting generative composition (where programs can generate new instruments and scores algorithmically), graphical tools (for programmers and non-programmers alike to collaborate), and new modes of social programming that allow musicians to easily adapt, extend, and rework existing instruments without having to ``fork" their code.

In contrast to other emerging web audio libraries and tools, Flocking provides a robust, optimized, and well-tested architecture that explicitly supports extensibility and long-term growth. Flocking runs in nearly any modern JavaScript environment, including desktop and mobile browsers (Chrome, Firefox, and Safari) as on embedded devices with Node.js

\end{abstract}
%

\section{Introduction}\label{sec:introduction}

Many computer music programming languages have been developed to enable musicians, composers, artists, students, and explorers to engage music through code. Each language solves a particular problem of efficiency, expressivity, or portability. Many aim to make advances in all three areas - Flocking is a new computer music programming language that aims to achieve exactly this.

This paper will demonstrate Flocking's expressivity with a discussion of the declarative nature paradigm utilized, efficiency with a discussion of optimization tactics for javascript webaudio languages and a comparison with other languages, and portability with a discussion of the underlying framework and Flocking's ability to communicate with other tools (GUI, MIDI, OSC, etc).


\begin{verbatim}
var synth = flock.synth({
    synthDef: {
        ugen: "flock.ugen.sinOsc",
        freq: 440,
        mul: 0.25
    }
});

\end{verbatim}


\section{The Framework}

\subsection{Declarative Programming}
We described Flocking above as a declarative framework, and this characteristic is essential to understanding its design. Though clear, unambiguous definitions of the technique are scarce both in the academic literature (Lloyd 2) and the forums of practicing programmers (C2 Wiki), declarative programming can be understood in the context of Flocking as having two essential characteristics:
it emphasizes a high-level, semantic view of a program’s logic and structure
it represents programs as data structures that can be understood by other programs

J.W. Lloyd's informal definition of declarative programming is probably the most helpful starting point for establishing a pragmatic understanding of the first point. ``Declarative programming involves stating {\it what} is to be computed but not necessarily {\it how} it is to be computed" (1). The emphasis here is on the logical or semantic aspects of computation, rather than on low-level sequencing and control flow. For musicians, the familiar unit generator abstraction established in the 1950s by Max Mathews in Music IV (34) offers at least a partial example of the declarative approach. Unit generators represent a higher-level abstraction for audio signals; they're building blocks that a musician specifies and connects together. The synthesis engine takes care of producing a working instrument from this “orchestra” specification, abstracting away the process of actually producing the samples for output. In this model, the musician doesn't directly specify the line-by-line sequencing of {\it how} a sine wave, for example, is produced. Rather, she specifies the desired signal-generating processes by name and lets the synthesis environment do the rest.

The second point, relating to the use of common data structures to represent programs, is the more subtle and also more significant aspect of Flocking’s declarative design. Traditional imperative programming styles are typically intended for an “audience of one”--the compiler. Though code is often shared amongst multiple developers, it can’t be understood or manipulated by a variety of other programs aside from the compiler.

In contrast, declarative programming involves the ability to write programs that are themselves represented in a format that can be processed by other programs as ordinary data. The Lisp family of languages are a well-known example of this approach. Paul Graham describes the declarative nature of Lisp, saying it ``has no syntax. You write programs in the parse trees... [that] are fully accessible to your programs. You can write programs that manipulate them... programs that write programs." While Flocking is written in the JavaScript programming language, it shares with Lisp the idea of expression programs within data structures (written as JSON), which are fully available for manipulation by other programs.

(example that further clarifies how this is the case in Flocking)

\subsection{JSON}

The key to Flocking's declarative approach is JSON, the JavaScript Object Notation format. JSON is a lightweight data interchange format based on a subset of JavaScript that can be parsed and understood in nearly any programming language (Crockford {\it json.org}). JSON provides two primary data structures that are practically universal across programming languages:

1. objects consisting of key/value pairs (i.e. dictionaries or associative arrays)
2. ordered lists (i.e. arrays or lists)

Since JSON's syntax and semantics are identical to the Object and Array type literals in JavaScript, JSON is an exceptionally convenient language for representing data in web applications. All of Flocking's musical primitives are expressed as trees of JSON objects. Where Max and PD provide compositionality by defining patch objects, and SuperCollider and ChucK provide this with functions, Flocking enables exceptional compositionality through ``documents"--declarative JSON specifications--and a simple document merging process enabled by the Infusion framework.

\subsection{Infusion}
Infusion, infusion, infusion, example of infusion.

(example of what an infused band of instruments looks like)

\section{Web Audio}

It is the opinion of the authors that the current W3C Web Audio API, without the support of additional libraries, is insufficient to support the expressivity required by creative musicians. Many of the limitations of the API are outlined in detail in (Wyse and Subramanian). Web Audio currently provides limited options for web developers who want to create their own custom synthesis algorithms in JavaScript and expect them to perform well. In particular, it is difficult to mix native and JavaScript-based nodes in the same signal graph without imposing latency and synchronization issues.

Flocking predates the first Web Audio API implementation, and was architected specifically for an environment where web developers could contribute their own first-class signal processing implementations. As a result of this philosophy, and due to the performance and ``developer experience" issues of the current Web Audio specific, Flocking uses only small parts of the API. Instead, it takes full control of the sample-generation process and provides musicians with an open palette of signal-generating building blocks that can be used to assemble sophisticated digital instruments.

Several other libraries also take a similar ``all JavaScript" approach. {\it Gibberish} (Roberts) and {\it CoffeeCollider} (Mohayonao) are two of the more prominent alternatives to Flocking. CoffeeCollider attempts to replicate the SuperCollider language as closely as possible using the CoffeeScript programming language (Ashkenas), while Gibberish takes a more traditional object-oriented approach. While these environments each offer their own unique features, none have attempted to stray far from the conventional models established by existing music programming environments.

Flocking, too, has taken architectural inspiration from a variety of existing music programming environments, particularly the design of the SuperCollider 3 synthesis server. Flocking shares with scsynth a simple ``functions and state" architecture for unit generators, as well as a strict (conceptual) separation between the strict realtime constraints of the signal-processing world and the more dynamic and event-driven application space (McCartney 64).

Flocking, however, takes a markedly different approach to the semantic and constructional issues of digital instrument-building than SuperCollider or other imperative object-oriented environments. In several key ways, it is inspired by the web architecture itself, particularly the emphasis on declarative programming (Berners-Lee x) and stable, publicly addressable state (Fielding x).

\subsection{Performance}

Much has been written about the current performance issues related to the current generation of JavaScript runtimes generally (lack of deterministic garbage collection) and the Web Audio API specifically (the requirement for ScriptProcessorNodes to run on the main browser thread) [Wyse and Subramanian] [Roberts]. It appears likely that the performance characteristics of the JavaScript language in general will continue to improve as the browser performance wars continue to rage between Mozilla, Google, and Apple. In addition, Web Worker-based strategies for sample generation are currently being discussed for inclusion in the Web Audio API specification (https://github.com/WebAudio/web-audio-api/issues/113).

In the interim, many claims have been made about the relative performance merits of various optimization strategies used in toolkits such as Gibberish (Roberts x). Most of these claims, however, focus on micro-benchmarks that measure only the cost of small-scale operations in isolation, rather than taking into account the performance of real-world signal processing algorithms.

In contrast to the risks of micro-benchmarking and premature optimization, the approach we have taken in Flocking is to focus on an architecture and framework that can serve as a flexible, long-term foundation on which to continually evolve new features and improved performance. Significant effort has been invested into developing automated unit and performance tests for Flocking that measure the real-world costs of its approach.

With just-in-time compilers such as Google's V8 and Mozilla's IonMonkey, we believe that real-world performance is best achieved by using simple algorithms that represent stable ``hot loops" that can be quickly and permanently be compiled into machine code by the runtime. The risk of micro-optimization attempts such as the code-generation techniques promoted by Gibberish is ``lumpy" (i.e. indeterminate) real-world performance. This risk is caused by the JavaScript runtime having to re-trace and recompile code that is dynamically generated and evaluated whenever the signal generating graph changes. Flocking avoids this risk while maintaining competitive performance by using a simple ordered list scheme for traversing and evaluating unit generators.


%\section{Floats and equations}

%\subsection{Equations}
%Equations should be placed on separated lines and numbered.
%The number should be on the right side, in parentheses.
%\begin{equation}
%E=mc^{2}.
%\label{eq:Emc2}
%\end{equation}

%\pagebreak

%\subsection{Figures, Tables and Captions}
%All artwork must be centered, neat, clean, and legible. All lines should be very dark for purposes of reproduction and artwork should not be hand-drawn. The proceedings will be distributed in electronic form only, therefore color figures are allowed. However, you may want to check that your figures are understandable even if they are printed in black-and-white.
%\begin{table}[h]
% \begin{center}
% \begin{tabular}{|l|l|}
%  \hline
%  String value & Numeric value \\
%  \hline
%  Hello ICMC SMC  & 2014 \\
%  \hline
% \end{tabular}
%\end{center}
% \caption{Table captions should be placed below the table.}
% \label{tab:example}
%\end{table}

%Numbers and captions of figures and tables always appear below the figure/table. Leave 1 line space between the figure or table and the caption. Figure and tables are numbered consecutively. Captions should be Times 10~pt. Place tables/figures in text as close to the reference as possible, and preferably at the top of the page.

%\begin{figure}[h]
%\centering
%\includegraphics[width=0.9\columnwidth]{Athens_University_main_building.pdf}
%\caption{Figure captions should be placed below the figure,
%exactly like this.
%This photo shows the main building of the University of Athens.\label{fig:example}}
%\end{figure}

%Always refer to tables and figures in the main text, for example:
%see Figure \ref{fig:example} and \tabref{tab:example}.

%\begin{figure}[t]
%\figbox{
%\subfloat[][]{\includegraphics[width=60mm]{figure}\label{fig:subfigex_a}}\\
%\subfloat[][]{\includegraphics[width=80mm]{figure}\label{fig:subfigex_b}}
%}
%\caption{Here's an example using the subfig package.\label{fig:subfigex} }
%\end{figure}



\section{Conclusions}
Awesomeness has been demonstrated.

\nocite{*} % just to put everything in reference section for now

\begin{acknowledgments}
Funding agencies? Pets?
\end{acknowledgments}

%%%%%%%%%%%%%%%%%%%%%%%%%%%%%%%%%%%%%%%%%%%%%%%%%%%%%%%%%%%%%%%%%%%%%%%%%%%%%
%bibliography here
\bibliography{flockingbibliography}

\end{document}
